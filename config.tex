\usepackage{lipsum}
\usepackage{microtype}
\usepackage{blindtext}

%
% FONTS
% 
\usepackage{ETbb}
\usepackage[T1]{fontenc}

%
% PAPERSIZE AND MARGINS
% see page 1 of the memoir manual
%
\setstocksize{8in}{5in} % size of the paper
\settrimmedsize{\stockheight}{\stockwidth}{*}

%%\setbinding{<length>}
\setlrmarginsandblock{0.5in}{*}{1.5} 
\setulmarginsandblock{0.75in}{*}{1.5} 
\checkandfixthelayout

%
% PAGE STYLE
%
%
% Custom page style called "mypagestyle"
%
\makepagestyle{mypagestyle} % Create a new pagestyle
\makepsmarks{mypagestyle}{%
	% This is where we specify what \leftmark and \rightmark contains
	\createmark{chapter}{right}{nonumber}{}{\quad}
	%createmark{chapter}{left}{nonumber}{}{\quad} % the left side has the same text always (title), so we don't need this	
	
	% Might want to keep these, see the manual for further information.
	\createplainmark{toc}{both}{\contentsname}
	\createplainmark{lof}{both}{\listfigurename}
	\createplainmark{lot}{both}{\listtablename}
	\createplainmark{bib}{both}{\bibname}
	\createplainmark{index}{both}{\indexname}
	\createplainmark{glossary}{both}{\glossaryname}}

% Here's where you style the header
\makeevenhead{mypagestyle}% Even page = left page
{\normalfont\thepage\quad\sffamily\bfseries\MakeUppercase\thetitle}{}{}
\makeoddhead{mypagestyle}% Odd page = right page
{}{}{\normalfont\sffamily\bfseries\rightmark\quad\normalfont\thepage}

% Activate the new pagestyle
\pagestyle{mypagestyle}


%%
%% CHAPTER STYLE
%%
%\makechapterstyle{mychapterstyle}{% Default values:
%  \setlength{\beforechapskip}{50pt}
%  \renewcommand*{\chapterheadstart}{\vspace*{\beforechapskip}}
%  \renewcommand*{\chapnamefont}{\normalfont\huge\bfseries}
%  \renewcommand*{\printchaptername}{\chapnamefont \@chapapp}
%  \renewcommand*{\chapternamenum}{\space}
%  \renewcommand*{\chapnumfont}{\normalfont\huge\bfseries}
%  \renewcommand*{\printchapternum}{\chapnumfont \thechapter}
%  \setlength{\midchapskip}{20pt}
%  \renewcommand*{\afterchapternum}{\par\nobreak\vskip \midchapskip}
%  \renewcommand*{\printchapternonum}{}
%  \renewcommand*{\chaptitlefont}{\normalfont\Huge\bfseries}
%  \renewcommand*{\printchaptertitle##1}{\chaptitlefont ##1}
%  \setlength{\afterchapskip}{40pt}
%  \renewcommand*{\afterchaptertitle}{\par\nobreak\vskip \afterchapskip}
%}
%\chapterstyle{mychapterstyle}


%%
%% LOWER SECTIONS STYLE
%%

%% Section
%\setbeforesecskip{-3.5ex plus -1ex minus -.2ex} %If the values are negative then the first line after the heading will not be indented.
%\setsecindent{0em}
%\setsecheadstyle{\Large\bfseries}
%\setaftersecskip{2.3ex plus .2ex} %If the value is negative, then the heading will be run-in

%% Subsection
%\setbeforesubsecskip{-3.25ex plus -1ex minus -.2ex} %If the values are negative then the first line after the heading will not be indented.
%\setsubsecindent{0em}
%\setsubsecheadstyle{\large\bfseries}
%\setaftersubsecskip{1.5ex plus .2ex} %If the value is negative, then the heading will be run-in

%% Subsubsection
%\setbeforesubsubsecskip{-3.25ex plus -1ex minus -.2ex} %If the values are negative then the first line after the heading will not be indented.
%\setsubsubsecindent{0em}
%\setsubsubsecheadstyle{\bfseries}
%\setaftersubsubsecskip{1.5ex plus .2ex} %If the value is negative, then the heading will be run-in

%% Paragraph
%\setbeforeparaskip{3.25ex plus 1ex minus .2ex}
%\setparaindent{0em}
%\setparaheadstyle{\bfseries}
%\setafterparaskip{-1em} %If the value is negative, then the heading will be run-in

%% Subparagraph
%\setbeforesubparaskip{3.25ex plus 1ex minus .2ex}
%\setsubparaindent{\parindent}
%\setsubparaheadstyle{\bfseries}
%\setaftersubparaskip{-1em} %If the value is negative, then the heading will be run-in

%%
%% LETTRINE CONFIG
%%

\usepackage{lettrine}
\usepackage{Zallman} %https://ftp.acc.umu.se/mirror/CTAN/macros/latex/contrib/lettrine/doc/lettrine.pdf
% https://ftp.acc.umu.se/mirror/CTAN/macros/latex/contrib/cfr-initials/cfr-initials.pdf
%
\renewcommand{\LettrineTextFont}{\rmfamily}
\LettrineRealHeighttrue
%\newcommand{\mylettr}[1]{\lettrine{\zall{#1}}} % lettrine with initial letter
%\newcommand{\mylettr}[1]{#1} % Turn off lettrine
\newcommand{\mylettr}[1]{\lettrine[lines=2]{#1}} % lettrine without initial letter


%% 
%% see page  of the memoir manual
%%


%% 
%% see page  of the memoir manual
%%


%% 
%% see page  of the memoir manual
%%